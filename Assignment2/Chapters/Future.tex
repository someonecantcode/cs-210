\section{Future potentials}

\subsection{myBusFare improvements}
Usually without creating specific person struct for the with names and their specific BusFare, you'll get something like this.
It's dry and honestly redundant. It would be better if we could have an array. This is the bare minimum solution.

\lstinputlisting[language=Java]{Chapters/v1myBusFare.java}
\ \\
This is what we get but with two arrays with same elements correlating to the same object, you can see some improvement.
Later on you can improve upon this but creating a struct. 

\lstinputlisting[language=Java]{Chapters/v2myBusFare.java}

Afterwards, you can should instead make the person a seperate file.


\lstinputlisting[language=Java]{Chapters/v3myBusFare.java}

This is where we arrive upon with extending classes. 
\begin{figure}[H]
        \frame{\includegraphics[width=1\textwidth,height=1\textheight,keepaspectratio]{Objects.png}}
        \caption[short]{Inheritance Structure with new object types}
\end{figure}


\subsection{persons}
This \texttt{persons} class utilizes ArrayList so that people in the future can easily add more passengers. This class
gives us the ability to clearly see who our people and basically have full control. 

\lstinputlisting[language=Java]{persons.java}

\subsection{person}
This \texttt{person} class handles behavior for individual people to accodomoate for their \texttt{name} and \texttt{BusFare} type.
It uses runtime polymorphism so that we can call the output() method in every \texttt{BusFare} object which makes it very
efficient. It's almost like a struct, providing us with framework of fields and methods to display.

\lstinputlisting[language=Java]{person.java}