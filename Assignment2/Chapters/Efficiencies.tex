\section{Efficiencies}

When we first were programming this assignment, it was important for it to follow a specific structure with 
super classes and subclasses inheriting methods. This would allow us to reuse code and override them to create our
own behaviors and additions. In the end, we are able to create instances and run their methods for our purposes: specifically 
looking at \texttt{Chloe (Adult)}, \texttt{Ted (Senior)}, \texttt{Ed (SuperSenior)} fares. This is the whole idea of the OOP paradigm, allowing
us to extend our code for many different purposes. 

\begin{figure}[H]
        \frame{\includegraphics[width=0.8\textwidth,height=0.8\textheight,keepaspectratio]{Inheritance.png}}
        \caption[short]{Inheritance Structure }
\end{figure}

\subsection{myBusFare}

In the end, this is how how our main myBusFare.java performs.  We can simply create our instances and run its output method.
This makes our code extremely simply with abstractions. Additionally, in the future, if anyone creates instances,
we can update the value of our baseRate and it will cascade and be updated.

\lstinputlisting[language=Java]{myBusFare.java}

\subsection{BusFare}
Let's take a look at the super class BusFare.java and uncover how it is efficient. In this
code snippet, we can encapsulate our class fields \texttt{baseRate} and \texttt{color} with our getter methods. All 
of these methods provide a structure for of all subclasses to inherit. The most efficient
part of the code is the \texttt{fare(String name)} method as allows us to just write one singular method 
to output our instance data since subclasses have their own overriden methods.
\ \\
\lstinputlisting[language=Java]{BusFare.java}

\newpage
\subsection{Adult}
This \texttt{Adult} subclass inherits all of the super methods such as \texttt{getBaseRate()} and \texttt{getColor()}. We
can overrides these methods, \texttt{getFare()} \& \texttt{getColor()}, with our own class behavior.
 \texttt{getColor()} again is basically one-to-one because color is static to the class. 
\ |\
\lstinputlisting[language=Java]{Adult.java}

\ \\
\subsection{Senior}
This \texttt{Senior} subclass is basically the same as Adult and inherits all of the super methods such as \texttt{getBaseRate()} and \texttt{getColor()}. We
can overrides these methods, \texttt{getFare()} \& \texttt{getColor()}, with our own class behavior. 
\texttt{getColor()} again is basically one-to-one because color is static to the class. 
\ \\
\lstinputlisting[language=Java]{Senior.java}

\newpage
\subsection{SuperSenior}
This \texttt{SuperSenior} subclass extends \texttt{Senior} and inherits all of the super methods such as \texttt{getBaseRate()} and \texttt{getColor()}. We
can overrides these methods, \texttt{getFare()} \& \texttt{getColor()}, with our own class behavior. Something special now is the way we can
calculate our \texttt{getFare}. Instead of grabbing \texttt{BusFare} base rate and applying a discount, we can instead just take 50\% of the
\texttt{Senior} fare cost. \texttt{getColor()} again is basically one-to-one because color is static to the class. 

\lstinputlisting[language=Java]{SuperSenior.java}


